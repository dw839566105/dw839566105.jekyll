\documentclass[UTF8]{ctexart}
\usepackage{ctex}
\usepackage[colorlinks,linkcolor=red]{hyperref}
\title{博资考复习}
\author{DW}
\begin{document}
\maketitle
\newpage
\tableofcontents
\newpage
\section{力学}
	\subsection{质点运动学}
		\subsubsection{质点和参考系}
			\par 质点突出了''物体具有质量''、''物体占有位置''。
		\subsubsection{速度和加速度}
			\par 速度:
			\begin{equation}
				\vec{v}(t) = \lim_{\Delta_t \to 0} \frac{\vec{r}(t+\Delta t)-\vec{r}(t)}{\Delta{t}}= \frac{\mathrm{d}\vec{r}}{\mathrm{d}t}
			\end{equation}
			\par 加速度:
			\begin{equation}
				\vec{a}(t) = \lim_{\Delta_t \to 0} \frac{\vec{v}(t+\Delta t)-\vec{v}(t)}{\Delta{t}}= \frac{\mathrm{d}\vec{v}}{\mathrm{d}t} = \frac{\mathrm{d}^2\vec{r}}{\mathrm{d}t^2}
			\end{equation}
		\subsubsection{直角坐标系中运动的描述}
			\par 把位置矢量按照直角坐标分解为$x$,$y$和$z$方向。
		\subsubsection{自然坐标系中运动的描述}
			\begin{itemize}
				\item 切向加速度和法向加速度\\
					\begin{equation}
						a_t = \vec{a}\cdot \hat{v} = \frac{\mathrm{d}v(t)}{\mathrm{d}t}
					\end{equation}				
					\begin{equation}
						a_n = \vec{a}\cdot \hat{n} = \frac{v^2(t)}{\vec{R}(t)}
					\end{equation}
					\begin{equation}
						a(t) = \sqrt{\big(\frac{\mathrm{d}^2s}{\mathrm{d}t^2}\big)^2+\big(\frac{1}{R}\big(\frac{\mathrm{d}s}{\mathrm{d}t}\big)^2\big)^2}
					\end{equation}
					\par 假设轨迹一小段为圆弧,$\rho$为曲率半径。其中$a_t$表示了质点速率随时间的变化率,$a_n$则反映了质点运动方向变化的快慢。
					\par 轨迹上任意一点的曲率半径:
					\begin{equation}
						R(t) = \frac{v^3(t)}{|\vec{a}(t)\times\vec{v}(t)|}
					\end{equation}
				\item 自然坐标系
					\par 加入第三个轴 $\hat{e}_3 = \vec{v} \times \vec{n}$
				\item 圆周运动
					\begin{equation}
						a_t = \vec{a}\cdot \hat{v} = \frac{\mathrm{d}v(t)}{\mathrm{d}t}
					\end{equation}				
					\begin{equation}
						a_n = \vec{a}\cdot \hat{n} = \frac{v^2(t)}{\vec{R}(t)}
					\end{equation}
					\par 如果是匀速圆周运动第一项$a_t$等于0。
					\par 定义角速度矢量$\vec{\omega}$,大小为$\frac{\mathrm{d}\theta}{\mathrm{d}t}$
					\begin{equation}
						\vec{v} = \vec{\omega}\times\vec{r}
					\end{equation}
					\begin{equation}
						\vec{a} = \frac{\mathrm{d}\omega}{\mathrm{d}t}\times\vec{r}+\omega\times(\omega\times\vec{r})
					\end{equation}
			\end{itemize}
		\subsubsection{平面极坐标系中的运动描述}
			\begin{equation}
				\vec{v}_r = \dot{r}\hat{r}
			\end{equation}
			\begin{equation}
				\vec{v}_r = \dot{r}\hat{r}
			\end{equation}
			\begin{equation}
				\vec{v}_{\theta} = r\dot{\theta}\hat{\theta}
			\end{equation}
			\par 利用
			\begin{equation}
				\frac{\mathrm{d}\hat{r}}{\mathrm{d}\theta} = \hat{\theta}
			\end{equation}
			\begin{equation}
				\frac{\mathrm{d}\hat{\theta}}{\mathrm{d}\theta} = \hat{r}
			\end{equation}
			\par 可以得到
			\begin{equation}
				\vec{a}_r = (\ddot{r}-r\dot{\theta^2})\hat{r}
			\end{equation}
			\begin{equation}
				\vec{a}_\theta = (2\dot{r}{\theta}+r\ddot{\theta})\hat{\theta}
			\end{equation}

	\subsection{质点运动学}
		\subsubsection{牛顿运动定律}
			\begin{itemize}
				\item 牛顿第一定律(惯性定律)
					\par 每个物体都保持精致火匀速直线运动的状态,除非有外力作用于他迫使它改变那个状态。
				\item 牛顿第二定律
					\par 运动的变化正比于外力,变化的方向沿外力作用的直线方向。
					\begin{equation}
						\vec{F} = m\vec{a} \Rightarrow m = \vec{F}/\vec{a}
					\end{equation}
				\item 牛顿第三定律(作用力与反作用力定律)
					\par 每一种作用都有一个相等的反作用;或者两个物体间的相互作用总是相等的。在同一条直线上,而且方向相反。
			\end{itemize}
			\par 只有第三个不需要惯性系的限制。
		\subsubsection{常见的力}
			\begin{itemize}
				\item 弹性力
					\begin{equation}
						\vec{F} = -k\vec{x}
					\end{equation}
				\item 摩擦力
					\begin{enumerate}
						\item 干摩擦
							\par 对于动摩擦
							\begin{equation}
								f_k = \mu_k N
							\end{equation}
							\par 对于静摩擦
							\begin{equation}
								f_s \leq \mu_s N
							\end{equation}
						\item 湿摩擦
							\par 相对运动不大
							\begin{equation}
								\vec{F} = \eta\vec{v}
							\end{equation}
							\par 相对运动大
							\begin{equation}
								F = \eta v^2
							\end{equation}
					\end{enumerate}
				\item 重力
					\begin{equation}
						F = mg
					\end{equation}
				\item 万有引力
					\begin{equation}
						F = G\frac{m_1m_2}{r^2}
					\end{equation}
				\item 库仑力
					\begin{equation}
						F = k\frac{q_1q_2}{r^2}
					\end{equation}
				\item *分子力
				\item *核力
				\item 洛伦兹力
					\begin{equation}
						\vec{F} = q\vec{v}\times\vec{B}
					\end{equation}
			\end{itemize}
	\subsection{非惯性参考系}
		\subsubsection{非惯性参考系与虚拟力}
			\begin{itemize}
				\item 平动参考系
					\begin{equation}
						\vec{F}_{eff} = m\vec{a}'
					\end{equation} 
				\item 转动参考系
					\begin{equation}
						\frac{\mathrm{D}\vec{r}'}{\mathrm{D}t} = \frac{\mathrm{d}\vec{r}'}{\mathrm{d}t} + \vec{\omega}\times \vec{r}'
					\end{equation} 
					\begin{equation}
						\frac{\mathrm{D}\vec{v}'}{\mathrm{D}t} = \frac{\mathrm{d}\vec{v}'}{\mathrm{d}t} + \vec{\omega}\times \vec{v}'
					\end{equation}
					\par 若两个系原点相对静止,可得惯性离心力
					\begin{equation}
						\vec{f}_i = -m\vec{\omega}\times(\vec{\omega}\times \vec{r}')
					\end{equation}
					\par 若相对于转动参考系还做匀速运动,物体还受科里奥利力(方程30)
					\begin{equation}
						\vec{f}_{cor} = -2m\vec{\omega}\times\vec{v}'
					\end{equation}
			\end{itemize}

			
\end{document}